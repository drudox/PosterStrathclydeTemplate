\chapter{Future work plan}\label{chap_futurework}
% A future work plan should be provided to identify new research questions based on your findings. At this point you may not have attempted all of your research objectives and so the outstanding questions should also be stated. You should include a Gantt chart detailing a timeline of when each task will be undertaken.

% ``...should include a Gantt chart..''... see appendix

For all the new study that we plan for the future a literature review is taken into account. Hence I will not remarks it for each year plans

\section{First Year (end of)}


\subsection{Multi-Chain Polymer Solution (stretching)}
Until the end of the first year we would like to have enough result about the stretching events, in order to fully understand this phenomena and have the materials (results of simulation) to submit a journal-paper, in order to have enough result we plan to proced in this way:

\begin{itemize}


\item Continuing to investigate the stretching for a single chain
\item Moving to investigate a solution with multi-chain (semidilute) solution.
\item Looking for ``entanglement'' phenomena (to see how the ``entanglement'' between polymer-chains affects the phenomena in terms of magnitude and number of events) 
\item Preparing a scientific paper to be published
\item Obtaining accurate knowledge of how the turbulent eddies interact with the polymer-chain is required in this framework, and in order to do this, continue the simulation with single chain using different condition. 


\end{itemize}

The only problem that we can come across during this step is the large time required for the solution of an accurate Direct numerical Simulation coupled with the langevin equation, since the code is not yet 
paralellized and run only into one core for simulation.




\section{Second Year}

\subsection{Polymer in Extensional Flow}

For the second year of this project our aims is to study the system (polymer-solvent) in ``Extensional Flow'' regime, and be able to predict the stress level of the chain (and chains) in this new regime (in extensional flow) . 

This involve:
\begin{itemize}

\item New developments in the Fortran90 code, that can require almost 4 months (or more)

\item Once the code is ready we can running simulation with single and multiple polymer chains with the goal to be able to evaluate:
\begin{itemize}
  \item The difference between single chain solution and multi chain solution

  \item See how the presence of entanglments between chains influence the solution behaviour.
\end{itemize}

Our aim for the second years is to publish in scientific journal about Polymer behaviour in Extensional Flow. 

Prepare some partial results in order to be presented at the conference of the \emph{British Applied Mathematics Colloquium 2019} (March or April) 

At the end of the second year of Ph.D we plan to starting writing the PhD thesis the first 2 chapter of the Ph.D thesis. 


\end{itemize}
\section{Third Year}

\subsection{Polymer in Shear Flow}
For the third year of Ph.D we plan to do in parallel with the thesis developing, the study of \textbf{Polymer In Shear Flow} and publish this study in a scientific journal.  

In order to study the shear flow, some step are required :

\begin{itemize}
    \item  Modify the Fortran code, implement the new boundary condition required for shear flow (this required almost 4 months)
    \item  Once have the code ready to run shear flow simulations we would like to study: 
        \begin{enumerate}
            \item A series of calculation with different concentrations of polymer and shear level
            \item Misure the elastic stress of the polymer chain and compare it with experimental result 
            \item How the polymer chains arrange into the shear flow 
        \end{enumerate} 
    \item Partecipate as presenter in the scottish conference of Fluid Mechanics
\end{itemize}

\section{Not planned}
Not planned officially but still thing that we can do if able:
\begin{itemize}
    \item Making the code parallizable (usable in more than one core) 
    \item Implement the wall boundary condition 
\end{itemize}    




