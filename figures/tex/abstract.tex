\chapter*{Abstract}
\addcontentsline{toc}{chapter}{Abstract}
%The abstract should be an overall summary for conducting the study and should include the objectives, methods used (not in great detail), major results or findings and conclusions. The abstract should be no longer that 1 page in length, therefore should be brief but informative and concisely written. The abstract should not contain new information (i.e. information not contained elsewhere in the report), references, acronyms, abbreviations, tables, or figures. Tip: it’s often easier to write the abstract last.

%Much of human history has been influenced by the availability of materials. 
%In facts, history is divided into eras named after the primary materials used; the Stone age, the Bronze age, and the Iron age.
%Similary, we can assert that in the twentieth century we entered the \textit{Polymer Age}.~\cite{pp}
%The research about the polymers in the last decade is increase exponentially; the behaviour 
%of this \textit{macromolecules} are not yet completely know, in particular the study from the mesoscopic point of view, and the interations with external force.
%For this purpose the study of the polymeric liquids (interaction macromolecule - fluid) is the more suittable, and looks to be the best candidate for this kind of research, that involves in particular the rheologic state of the polymer and it's response at the external force due to the turbulent velocity field of the solvent.

%The study of the polymeric liquids is made in two different way :
%\begin{itemize}
%\item Experimentally : difficulty to catch the correct behaviour of the polymer chain.
%\item Numeric : using high order mesoscopic Computational Fluid Dynamics.
%\end{itemize}

%The purpose of this ph.D is to investigate the polymer chains response to the turbulence field surrounding by mean of Fortran90 high order uncompressible advanced CFD code that consider a mesoscopic point of view.
%This computational simulation are performed resolving the Navier Stokes \textit{Direct Numerical Simulation} to resolving the turbulent flow field of the solvent, coupled with the Langevin equations for the Polymer chain.
Nowadays the increasing use of polymers and the growing number of studies let us assert that the twentieth %twenty-first
century can be called the \textit{Polymer Age}.
The deeper understanding of chain behaviour has been possible by means of experimental and theoretical investigations and only in recent times new numerical strategies are available to this aim; Computational Fluid Dynamic is one of the most exploited numerical method, due to the smaller cost if compared with experimental one.
Until now a large amount of numerical studies approximated the chain behaviour with the well known bead-spring model or spring-spring model, for a specific kind of polymer or applying that in the biologic field: DNA long chain gained particular attention.

Most of the existing investigations aim to model the complex phenomenon of chain stretching, trying to find out theoretical correlations with the physical characteristics of polymers.

All numerical studies are associated by some semplifications in the chain behaviour approximation: authors focused their attention on hydrodynamic interactions and thermal fluctuation, neglecting chain entanglement in the majority of works; only in recent works the Brownian motion has been taken into account (Kivotides \emph{et al}~\cite{demos2}) and tried to find out a correlations in mesoscopic viscolelastic fluids.

Different analytical model have been developed to investigate molecules deformation excluding the volume interaction effects and an interesting contribution is rappresented by the work of Kivotides \emph{et al}~\cite{demos3}, in which authors proportionally linked stretching entity and steeper spectra.

In this work will be discussed the numerical approach used to model chain stratching: after the sections dedicated to the theory behind the polymer behaviour, the methodology used in the first year of PhD will be discussed.

The results exposed in this report reppresent a first step to the deep undersanding of molecules deformations, due to the huge effort of handling the complex home-made Fortran code. This needs an appropriate period of time to raise confidence and reliability of results obtained.









